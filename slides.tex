\documentclass[14pt]{beamer}

\usepackage{amsmath, amssymb, amsthm}
\usepackage{amsfonts, amsxtra}
\usepackage[english,russian]{babel}
\usepackage[utf8]{inputenc}
\usepackage[T2A]{fontenc}

\usepackage{amsmath}
\usepackage{amssymb}
\usepackage{amsthm}
\usepackage{amsfonts}
\usepackage{amsxtra}

\setcounter{secnumdepth}{0}

%%%%%%%%%%%%%%%%%%%%%%%%%
\usepackage{array}
% \setlength\extrarowheight{5pt}
\renewcommand{\arraystretch}{2.3}
%%%%%%%%%%%%%%%%%%%%%%%%%
\usepackage{minted}
%%%%%%%%%%%%%%%%%%%%%%%%%
%\usepackage[top=2cm, bottom=2cm, left=1.5cm, right=2cm]{geometry}
\usepackage{parskip}
% \setlength{\parindent}{0cm}
%%%%%%%%%%%%%%%%%%%%%%%%
% \usepackage{forest} % for nice trees
%%%%%%%%%%%%%%%%%%%%%%%%
\usepackage{color, soul}
\newcommand{\hltexttt}[1]{\texthl{\texttt{#1}}}
\setlength{\fboxsep}{0pt}% 
%%%%%%%%%%%%%%%%%%%%%%%%
\listfiles
%%%%%%%%%%%%%%%%%%%%%%
\title{
}
\institute{Университет ИТМО}
\author{ Группа 5539. 
Демьянюк Виталий.
Кравцов Никита.
Рыбак Андрей.}
\date{}
\setbeamersize{text margin left=0.5cm,text margin right=0.5cm}
\begin{document}
{
\fontsize{14pt}{14pt}\selectfont
\setbeamertemplate{footline}{}
\begin{frame}
    \maketitle
\end{frame}
}

\begin{frame}
{Взаимная информация}
$$
I(X) = \sum\limits_{i=1}^{d} H(X_i) - H(X_1, \dots, X_d) 
$$
$
I(X_1, \dots, X_d) = 0  \iff $ independence

Энтропия
$$
H(X_1 \dots X_d) =
$$
$$
- \int f (x_1 \dots x_d) \log f (x_1 \dots x_d) d x_1 \dots d x_d
$$
\end{frame}
\begin{frame}
Информация Реньи
$$
I_{\alpha}  (X_1 \dots X_d)  = D_{\alpha}
\left ( f(x_1 \dots x_d) || \prod\limits_{i=1}^d f(x_i) \right ) =
$$
$$
= \frac{1}{1-\alpha} \log \int
\left ( \frac {\prod_{i=1}^d f(x_i)} {f(x_1 \dots x_d)} \right )^{\alpha}
f(x_1 \dots x_d) d x_1 \dots d x_d
$$
Энтропия Реньи
$$
H_{\alpha}  (X_1 \dots X_d) = \frac{1}{1-\alpha} \log \int
f^{\alpha} (x_1 \dots x_d)  dx_1 \dots dx_d
$$

$$
\lim_{\alpha \to 1} I_{\alpha} = I \quad \lim_{\alpha \to 1} H_{\alpha} = H
$$
\end{frame}

\begin{frame}
    Пусть $ \mathbf Z = (Z_1 \dots Z_d) = (g_1(X_1) \dots g_d(X_d)) = g(\mathbf X)  $, где 
    $ g_j : \mathbb R \to \mathbb R, j = 1 \dots d $ — монотонная функция

    Информация сохраняется после монотонных преобразований:

    $I_\alpha(\mathbf Z) = \int_{Z} \left ( \frac{f_{\mathbf Z}(\mathbf z)}{\prod_{j=1}^d f_{Z_j}(z_j)} \right )^\alpha
    \left ( \prod\limits_{j=1}^d f_{Z_j}(z_j) \right ) d {\mathbf z} = I_\alpha (\mathbf X)
    $
    Marginals of $\mathbf Z$ равномерны  $\implies I_\alpha (\mathbf Z) = -H_\alpha (\mathbf Z) $
\end{frame}

\begin{frame}
Copula transformation

$\mathbf X = [X_1 \dots X_d] \to [F_1(X_1) \dots F_d(X_d)] = [Z_1 \dots Z_d] = \mathbf Z$

$\implies I_\alpha (\mathbf X) = I_\alpha (\mathbf Z) = -H_\alpha (\mathbf Z) $

Свели задачу нахождения взаимной информации к задаче нахождения энтропии Реньи.

Но $F_i$ — неизвестны.

\pause

Решение: использовать эмпирические $F^n_j $ и эмпирическое копульное
преобразование.
\end{frame}

\begin{frame}
    \frametitle{Эмпирическое копульное преобразование}
$\mathbf X = ... $ 
\end{frame}
\begin{frame}
    \frametitle{Входные данные}
    \includegraphics[scale=0.5]{x.png}
    \includegraphics[scale=0.5]{y.png}
    \includegraphics[scale=0.5]{z.png}
\end{frame}
\begin{frame}

\end{frame}
\end{document}

